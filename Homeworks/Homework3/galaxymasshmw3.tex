\documentclass[12pt]{article} 
\usepackage{mathptmx}
\usepackage{newtxtext}
\usepackage{newtxmath}
\usepackage{graphicx}
\usepackage[a4paper, left=0.6in, right=0.6in, top=0.7in, bottom=0.7in]{geometry} 

\title{ASTR400B Galaxy mass table}
\author{Swapnaneel Dey}
\date{February 2025}

\begin{document}

\maketitle

\section{Table}

\begin{table}[ht!]
\centering
    \resizebox{1.0\textwidth}{!}{%
    \begin{tabular}{c|c|c|c|c|c}
    \hline
       Galaxy Name & Halo Mass ($10^{12}\,M_{\odot}$) & Disk Mass ($10^{12}\,M_{\odot}$) & Bulge Mass ($10^{12}\,M_{\odot}$) & Total ($10^{12}\,M_{\odot}$) & $f_\text{bar}$ \\
    \hline
        MW & 1.975 & 0.075 & 0.010 & 2.060 & 0.041 \\ 
        \hline
        M31 & 1.921 & 0.120 & 0.019 & 2.060 & 0.067 \\
        \hline
        M33 & 0.187 & 0.009 & 0 & 0.196 & 0.046 \\
        \hline
        Local Group &  & & & 4.316 & 0.054 \\
    \hline
    \end{tabular}}
    \caption{Component and total mass table for MW, M31, M33. Mass are in the units of $10^{12}\,M_{\odot}$.  }
    \label{tab:my_label}
\end{table}

1. How does the total mass of the MW and M31 compare in this simulation? What galaxy component dominates this total mass?

\textbf{Answer}: Both galaxies possess equal total mass, but their components vary in contribution. In both cases, Halo Mass dominates the contribution toward the total mass. 
\\

2. How does the stellar mass of the MW and M31 compare? Which galaxy do you expect to be more luminous?

\textbf{Answer:}The stellar mass of M31 is much more than MW. MW has stellar mass of 85$\times10^{9}\,M_{\odot}$ while M31 has 139$\times10^{9}\,M_{\odot}$. This suggests that the M31 is much more luminous since it consists of more stars, considering a similar stellar population.
\\

3. How does the total dark matter mass of MW and M31 compare in this simulation (ratio)? Is this surprising, given their difference in stellar mass?

\textbf{Answer:} $Dark Matter_{MW}/Dark Matter_{M31} \sim$ 1.03, pretty similar. This is surprising since it means that M31, while having a similar mass (mainly due to DM), has many more stars, meaning it could have recently gone in a star-forming burst, or maybe it has a better star-forming efficiency in general. 
\\

4. What is the ratio of stellar mass to total mass for each galaxy (i.e., the baryon fraction)? In the Universe, $\frac{\Omega_b}{\Omega_m} \approx 16\%$ of all mass is locked up in baryons (gas and stars) versus dark matter. How does this ratio compare to the baryon fraction you computed for each galaxy? Given that the total gas mass in the disks of these galaxies is negligible compared to the stellar mass, do you have any ideas for why the universal baryon fraction might differ from that in these galaxies?

\textbf{Answer:} The baryon fraction for each galaxy is given in the last column of Table \ref{tab:my_label}. These ratios are much less than the $f_\text{bar}$ = 0.16 for the universe. Maybe this is because the majority of the gas has been converted to stars and ejected from the galaxy itself upon its death. Or maybe the galaxies are not able to accrete gas outside of its halo very efficiently. Another possible explanation is that the universe in itself is not as dark matter dense as a galaxy would be, which has its total mass dominated by dark matter itself.


\end{document}
